\chapter{Chapter}
The following texts are instructions and examples on how to use this template.
This template is made by \href{https://github.com/sarphiv/}{github.com/sarphiv} via various examples from \href{https://tex.stackexchange.com}{tex.stackexchange.com}
and its structure is inspired by \rawref{Laursen's XeLaTeX thesis template}.
It is not advised to use this template as this is literally sarphiv's first LaTeX project - a mess and major bugs should therefore be expected.

\section{Section}
If chapters such as the preface, colophon, and/or acknowledgments are necessary
they can be uncommented in \lstinline|main.tex|.

The headings above and below are examples of the different depths that have been defined.

Go into the \rawref{settings.tex} file now and setup appropriate document metadata and properties.
A lot of the values in there are placeholders and should therefore never make it into the final draft.

\subsection{Subsection}
Below you can find the most commonly used files/locations.
\begin{itemize}
    \item \rawref{main.tex}
    \item \rawref{settings.tex}
    \item \rawref{bibliography.bib}
    \item \rawref{chapters/}
    \item \rawref{appendices/}
    \item \rawref{media/}
\end{itemize}

\subsubsection{Subsubsection}
The above list was generated with the \rawref{itemize} environment.
Use the \rawref{enumerate} environment to enumerate items.
Each item needs to be prefixed with \rawref{\textbackslash item XXXX}
where \rawref{XXXX} is the item.

\begin{enumerate}
    \item First
    \item Second item with enumerated subitems
    \begin{enumerate}
        \item First subitem
        \item Second subitem with a really long text that should break 
            this line at some point so we can see what that looks like
    \end{enumerate}
    \item Third item with itemized subitems
    \begin{itemize}
        \item Some item
        \item Some other item
    \end{itemize}
\end{enumerate}

An empty line break after \rawref{\textbackslash end} statement 
will cause the text after a list to be treated as a new paragraph.

Lists can have their label style changed by supplying optional argument e.g. \rawref{\lbrack label=\textbackslash roman*\rbrack} for enumerations.

\paragraph{Paragraph}
Arbitrary text can be generated with \rawref{\textbackslash lipsum[x][y]}
where \rawref{x} is the lorem ipsum paragraph (index or range) you want.
And where \rawref{y} is the sentence you want (index or range) 
e.g. \rawref{\textbackslash lipsum[1][1-7]}

\subparagraph{Subparagraph}
\lipsum[1][1-7]


\section{More examples}
The following are more examples and instructions on how to use various features.

\subsection{Text styles and families}
Text can have different styles and families.

\textit{The italic font style is used to hint at a different meaning}
and can be enabled with \rawref{\textbackslash textit\{\dots\}}.

\textbf{The bold font style is used for emphasis}
and can be enabled with \rawref{\textbackslash textbf\{\dots\}}.

\textsf{The sans serif family is used for document related markers}
and can be enabled with \rawref{\textbackslash textsf\{\dots\}}.

\texttt{The monospace (typewriter) family is used for code}\\
\texttt{and external references}
and can be enabled with \rawref{\textbackslash texttt\{\dots\}}.
The monospace font does not break at lines correctly when used this way.

\textmd{The normal font style (medium series) is the default font style}
and can be enabled with \rawref{\textbackslash textmd\{\dots\}}

\subsubsection{Enabling locally for an environment}
To locally enable a font style or family within an environment use \rawref{\textbackslash XXYY},
where \rawref{XX} is the two letter code for the style or family you want,
and where \rawref{YY} is either
\rawref{series} for styles
or \rawref{family} for  families.
An example could be \rawref{\textbackslash bfseries} for the bold style.


\subsection{Math
    \texorpdfstring
        {$\frac{\pi}{2} = \int_{-1}^{1} \sqrt{1-x^2}dx$}
        {equation pi calculated by integral}
}
Math in headings need to be surrounded by \rawref{\textbackslash texorpdfstring\{XXXX\}\{YYYY\}},
where \rawref{XXXX} is the math equation
and \rawref{YYYY} is the replacement to go into the PDF metadata table of contents.

Inline math can be input with \rawref{\$1+2\$}.
While standalone math equations can be input with the
\rawref{equation}, \rawref{multline}, \rawref{align}, and \rawref{gather} environments.
Adding an asterisk removes equation numbering e.g. \rawref{align*}.
Nest the \rawref{split} environment inside the others to group equations into one number.
\begin{equation}
    x = \mathcal{X}, \mathtt{x}, \mathbf{x}, \mathit{x},
        x_{1_{2}}^{1^{2}}
    \cdot var \times \text{string}
    %The '~' character is used to force the creation of a space
    %The '\!' character is used to force the creation of a small negative space
    ,~x \in \{y \in \mathbb{R}~|~y^2 = 0\}
\end{equation}
\begin{multline}
    p(x) = 3x^6 + 14x^5y + 590x^4y^2 + 19x^3y^3\\
        - 12x^2y^4 - 12xy^5 + 2y^6\\
        + 3x^6 + 14x^5y + 590x^4y^2
\end{multline}
\begin{align}
    %Split environment can be used to keep alignment
    % but while still grouping equation numbers
    \begin{split}
        \label{math-grouped-equation}
        x^{23} = ~&y + 97410\\
        0 = x - &y + \frac{c}{\sqrt{x}}
    \end{split}\\
    f(x) = \int_b^a c + &y dx, ~\text{aligned on } y
\end{align}
\begin{gather*}
    2x - 5y =  8 \\
    3x^2 + 9y =  3a + c \\
    P(\vec{k}) = \int_a^b e^{i\vec{k}\cdot\vec{R}} P(\vec{R}) d\vec{R}
\end{gather*}
Equations were displayed in the order introduced.
The \rawref{split} environment was nested for \cref{math-grouped-equation}.
The \rawref{gather*} environment was used for the last equations
to demonstrate unnumbered equations.
Beware of blank lines when formatting text and equations.


\subsection{Columns}
The body can be split into columns with the \rawref{multicols} environment.
This environment requires an argument specifying the amount of columns e.g.
\rawref{\textbackslash begin\{multicols\}\{2\}} for two columns.

An asterisk version of the environment also exists.
This version will disable a feature that makes both columns the same height
e.g. first column being 100\% of the page height and second column 50\% of the page height,
instead of \textbf{both} being 75\% of the page height.

\begin{multicols}{2}
    \lipsum[2]
\end{multicols}


\subsection{Notes}
Footnotes%
\footnote{Letters like æ, ø, å can be used in the document. }
can be made with \rawref{\textbackslash footnote\{\dots\}}.
Footnotes can also be really long%
\footnote{This is a really long footnote that will encounter a line break to show how footnotes wrap. The wrap should be happening about now}.
If the footnote source code is placed on a new line,
a comment mark \rawref{\%} should be placed at the end of the original line
to ensure the footnote reference is placed the close to the source.
The comment mark causes the new line character to be ignored.

\label{label-example} %References the above subsection - not this exact place

Sources %\footcite[23]{DUMMY:1}
can be cited with
\rawref{\textbackslash footcite[PPPP]\{XXXX\}}
where \rawref{PPPP} is the optional page number in the source,
and \rawref{XXXX} is the source key/label/ID%\footcite[11]{DUMMY:2}
.


\subsection{References}
Labels can be used to mark areas that can be referenced.
The label should be somewhere after the area that should be referenced e.g. a caption for a figure.
Labels can be placed with \rawref{\textbackslash label\{XXXX\}},
where \rawref{XXXX} is the label ID.
To reference a label such as \cref{label-example} use \rawref{\textbackslash cref\{XXXX\}},
where \rawref{XXXX} is the label ID.
The area type referenced will automatically be inferred.


\subsection{Markup}
External or inline code \rawref{references} can be made with 
\rawref{\textbackslash rawref\{XXXX\}} where \rawref{XXXX} is the reference. 
Brackets like \rawref{[} and \rawref{]} should \textbf{NOT} be escaped with '\textbackslash',
the brackets can be specified literally.

\begin{highlightbox*}{Highlighted box}
    Highlight boxes are also supported and can be accessed via the two environments
    \rawref{highlightbox} and \rawref{highlightbox*}.

    The asterisk version requires a title argument e.g.
    \rawref{\textbackslash begin\{highlightbox*\}\{XXXX\}}
    where \rawref{XXXX} is the title of the box.
\end{highlightbox*}

The highlight boxes can also be used to highlight equations.
\begin{highlightbox}
\begin{equation}
    1+2\neq 4
\end{equation}
\end{highlightbox}


\subsection{Figures}
Graphics can be included with code such as
\begin{figure}
    \centering
    %There should not be a space between the ']' and the '{',
    % it is there to allow an automatic linebreak
    \begin{lstlisting}
\begin{figure}
    \centering
    \includegraphics[width=0.9\textwidth] {media/missing-graphic-rectangle.pdf}

    \caption{Demonstration of missing graphic}
    \label{missing-graphic-rectangle-example}
\end{figure}
    \end{lstlisting}

    \caption{Code example for missing graphic example}
    \label{code-missing-graphic-rectangle-example}
\end{figure}


\begin{figure}
    \centering
    \includegraphics[width=0.9\textwidth]{media/missing-graphic-rectangle.pdf}

    \caption{Demonstration of missing graphic}
    \label{missing-graphic-rectangle-example}
\end{figure}

\begin{figure}
    \centering
    \begin{subfigure}{0.28\textwidth}
        \includegraphics[width=\textwidth] {media/missing-graphic-square.pdf}
        \caption{First subfigure}
    \end{subfigure}
    \begin{subfigure}{0.28\textwidth}
        \includegraphics[width=\textwidth] {media/missing-graphic-square.pdf}
        \caption{Second subfigure}
    \end{subfigure}
    \begin{subfigure}{0.28\textwidth}
        \includegraphics[width=\textwidth] {media/missing-graphic-square.pdf}
        \caption{Third subfigure}
    \end{subfigure}

    \begin{subfigure}{0.855\textwidth}
        \includegraphics[width=\textwidth] {media/missing-graphic-rectangle.pdf}
        \caption{Fourth subfigure}
    \end{subfigure}

    \caption{Demonstration of missing graphic subfigures with a really long caption that should cause a line break}
    \label{missing-graphic-subfigures-example}
\end{figure}

The output of the code in \cref{code-missing-graphic-rectangle-example}
is shown in \cref{missing-graphic-rectangle-example}.
Subfigures can be created via the example shown in \cref{code-missing-graphic-subfigures-example}.
The \rawref{\textbackslash label} can be omitted if unnecessary.
To create todo/(fix me) figures the right way, read \cref{fixmes-section}.

Code examples and graphics can be used without figures.
Although, figures help organize, reference, and describe media,
so please use them.


\subsection{Tables}
Tables can be created with the \rawref{longtable} environment.
This environment is used to enable tables spanning multiple pages if necessary.

%The second argument describes how many columns there are.
% The 'c' centers the first column,
% the 'l' left aligns the right column ('r' to right align),
% the 'S' aligns numbers by decimal point in the middle column
% the '|' creates a vertical line between two columns,
% and '||' creates a double vertical line between two columns,
% and 'p{NNNN}' where 'NNNN' is the width 
% vertically aligns text to the top to use with long text.
\begin{longtable}{c|S|l}
    %The new line is necessary when caption is at the top
    \caption{Small table example}\\
    %Label must be after caption to refer to table
    \label{small-table-example}

    \textbf{First} & \textbf{Second} & \textbf{Third}\\
    \hline
    \endfirsthead

    \textbf{First} & \textbf{Second} & \textbf{Third}\\
    \hline
    \endhead

    %Normal entry
    Alpha & 2.102 & USD\\

    %Multicolumn centered entry with right vertical line and horizontal lines
    \hline
    Bravo & \multicolumn{2}{c|}{NOTE}\\
    \hline

    %Multi-row entry where '*' specifies to auto-fit the width of the multiple rows
    Charlie & 213.21 & \multirow{2}{*}{ZERO}\\
    Delta & 1231\\
    
    %Colored cell with partial horizontal lines above and below
    \cline{2-3}
    Echo & \cellcolor{blue-base}\color{white-pure}132.193 & EUR\\
    \cline{1-2}

    %Multi-column, multi-row seems to need all this to ensure lines are drawn correctly
    \multicolumn{2}{c|}{\multirow{2}{*}{Foxtrot}} & \multirow{2}{*}{ETH}\\
    \multicolumn{2}{c|}{}\\

    %Double horizontal line to make a thicker horizontal line
    \hline
    \hline
    Golf & 12313.424719 & DOGE
\end{longtable}

As shown, captions can be placed at the top of figures. 
To see the source code for \cref{small-table-example} see \cref{code-small-table-example}.

%To color a column '>{\columncolor{XXXX}}', where 'XXXX' is the color,
% must be placed before the alignment letter for the column
\begin{longtable}{r|>{\columncolor{white-near}}l}
    \caption{Big table example}\\
    \label{big-table-example}

    %Everything before \endfirsthead is used as the first header
    \textbf{First} & \textbf{Second}\\
    \hline
    \endfirsthead

    %Everything after \endfirsthead but before \endhead
    % will be used as the headers for when the table reaches a new page.
    \textbf{First} & \textbf{Second}\\
    \hline
    \endhead
    
    1 & Alpha\\
    2 & Bravo\\
    3 & Charlie\\
    4 & Delta\\
    %To color a row '\rowcolor{XXXX}', where 'XXXX' is the color, must prefix the row.
    \rowcolor{yellow-dark}\color{white-pure}5 & Echo\\
    6 & Foxtrot\\
    7 & Golf\\
    8 & Hotel\\
    9 & India\\
    10 & Juliett\\
    11 & Kilo\\
    12 & Lima\\
    13 & Mike\\
    14 & November\\
    15 & Oscar\\
    16 & Quebec\\
    17 & Romeo\\
    18 & Sierra\\
    19 & Tango\\
    20 & Uniform\\
    21 & Victor\\
    22 & Whiskey\\
    23 & X-ray\\
    24 & Yankee\\
    25 & Zulu\\
\end{longtable}

The big table example \cref{big-table-example} demonstrates table spanning multiple pages.
To learn more about setting up headers for such a case, coloring columns, and coloring rows,
see \cref{code-big-table-example}.


\subsection{Todos/fixmes}
\label{fixmes-section}
In the settings file \rawref{settings.tex} fix me notes can be enabled/disabled.
Remember to disable in the final version of the document.

The first page is a list of fix me notes. 
A fix me can be placed
\fxnote{A note about something to be done}
with \rawref{\textbackslash fxnote\{XXXX\}} where \rawref{XXXX} is the text.

There are four different levels of severity for fix mes.
The \rawref{note} level is the lowest priority\fxnote{Note priority}.
The next level up is the \rawref{warning} level\fxwarning{Warning priority}.
The second highest level is the \rawref{error} level\fxerror{Error priority}.
The highest level is the \rawref{fatal} level\fxfatal{Fatal priority}, 
which makes compilation fail if present
and fix mes' state is set to \rawref{final} mode (a.k.a. \rawref{disabled}).

\begin{enumerate}
    \item \rawref{\textbackslash fxnote\{XXXX\}}
    \item \rawref{\textbackslash fxwarning\{XXXX\}}
    \item \rawref{\textbackslash fxerror\{XXXX\}}
    \item \rawref{\textbackslash fxfatal\{XXXX\}}
\end{enumerate}

A global fix me note%
\footnote{\label{all-severity-levels-supported}Other severity levels are also supported}
can be placed with 
\rawref{\textbackslash fxglobalnote\{XXXX\}} where \rawref{XXXX} is the text.
These notes should ideally be placed in the preamble.

\begin{figure}
    \centering
    \fxgraphicrectwarning[width=0.7\textwidth]{Add missing rectangle graphic}
    
    \caption{A missing rectangle graphic example}
\end{figure}

Code examples for placing missing figures can be seen at
\cref{code-missing-graphic-fix-me-example}.
Instead of using the 
\rawref{\textbackslash includegraphics[XXXX]\{ZZZZ\}} 
command, use either 
\rawref{\textbackslash fxgraphicrectnote[XXXX]\{YYYY\}}
or \rawref{\textbackslash fxgraphicsqnote[XXXX]\{YYYY\}},
where \rawref{XXXX} are options for the graphic such as width,
\rawref{YYYY} is the fix me text,
and \rawref{ZZZZ} is the graphic location which is not needed.
The point mentioned in \cref{all-severity-levels-supported} still stands.

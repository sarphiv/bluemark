\chapter{Code examples}
The following is a code example included as an appendix.
It is referenced just like any other reference described in \cref{label-example}.

\begin{figure}
    \centering
    Hello there
    \caption{Small text figure to show figure numbering in appendixes}
    \label{}
\end{figure}


\section{Code example for missing graphic subfigures example}
\label{code-missing-graphic-subfigures-example}
%There should not be a space between the ']' and the '{',
% it is there to allow an automatic linebreak in the source code showcase
\begin{lstlisting}
\begin{figure}
    \centering
    \begin{subfigure}{0.28\textwidth}
        \includegraphics[width=\textwidth] {media/missing-graphic-square.pdf}
        \caption{First subfigure}
    \end{subfigure}
    \begin{subfigure}{0.28\textwidth}
        \includegraphics[width=\textwidth] {media/missing-graphic-square.pdf}
        \caption{Second subfigure}
    \end{subfigure}
    \begin{subfigure}{0.28\textwidth}
        \includegraphics[width=\textwidth] {media/missing-graphic-square.pdf}
        \caption{Third subfigure}
    \end{subfigure}

    \begin{subfigure}{0.855\textwidth}
        \includegraphics[width=\textwidth] {media/missing-graphic-rectangle.pdf}
        \caption{Fourth subfigure}
    \end{subfigure}
    
    \caption{Demonstration of missing graphic subfigures  with a really long caption that should cause a line break}
    \label{missing-graphic-subfigures-example}
\end{figure}
\end{lstlisting}    


\section{Code for small table example}
\label{code-small-table-example}

\begin{lstlisting}[language=tex]
%The second argument describes how many columns there are.
% The 'c' centers the first column,
% the 'l' left aligns the right column ('r' to right align),
% the 'S' aligns numbers by decimal point in the middle column
% the '|' creates a vertical line between two columns,
% and '||' creates a double vertical line between two columns,
% and 'p{NNNN}' where 'NNNN' is the width 
% vertically aligns text to the top to use with long text.
\begin{longtable}{c|S|l}
    %The new line is necessary when caption is at the top
    \caption{Small table example}\\
    %Label must be after caption to refer to table
    \label{small-table-example}

    \textbf{First} & \textbf{Second} & \textbf{Third}\\
    \hline
    \endfirsthead

    \textbf{First} & \textbf{Second} & \textbf{Third}\\
    \hline
    \endhead

    %Normal entry
    Alpha & 2.102 & USD\\

    %Multicolumn centered entry with right vertical line and horizontal lines
    \hline
    Bravo & \multicolumn{2}{c|}{NOTE}\\
    \hline

    %Multi-row entry where '*' specifies to autofit the width of the multiple rows
    Charlie & 213.21 & \multirow{2}{*}{ZERO}\\
    Delta & 1231\\
    
    %Colored cell with partial horizontal lines above and below
    %(cell/row/column)color 
    \cline{2-3}
    Echo & \cellcolor{blue-base}\color{white-pure}132.193 & EUR\\
    \cline{1-2}

    %Multi-column, multi-row seems to need all this to ensure lines are drawn correctly
    \multicolumn{2}{c|}{\multirow{2}{*}{Foxtrot}} & \multirow{2}{*}{ETH}\\
    \multicolumn{2}{c|}{}\\

    %Double horizontal line to make a thicker horizontal line
    \hline
    \hline
    Hotel & 12313.424719 & DOGE
\end{longtable}
\end{lstlisting}


\section{Code for big table example}
\label{code-big-table-example}

\begin{lstlisting}[language=tex]
%To color a column '>{\columncolor{XXXX}}', where 'XXXX' is the color,
% must be placed before the alignment letter for the column
\begin{longtable}{r|>{\columncolor{white-near}}l}
    \caption{Big table example}\\
    \label{big-table-example}

    %Everything before \endfirsthead is used as the first header
    \textbf{First} & \textbf{Second}\\
    \hline
    \endfirsthead

    %Everything after \endfirsthead but before \endhead
    % will be used as the headers for when the table reaches a new page.
    \textbf{First} & \textbf{Second}\\
    \hline
    \endhead
    
    1 & Alpha\\
    2 & Bravo\\
    3 & Charlie\\
    4 & Delta\\
    %To color a row '\rowcolor{XXXX}', where 'XXXX' is the color, must prefix the row.
    \rowcolor{yellow-dark}\color{white-pure}5 & Echo\\
    6 & Foxtrot\\
    7 & Golf\\
    8 & Hotel\\
    9 & India\\
    10 & Juliett\\
    11 & Kilo\\
    12 & Lima\\
    13 & Mike\\
    14 & November\\
    15 & Oscar\\
    16 & Quebec\\
    17 & Romeo\\
    18 & Sierra\\
    19 & Tango\\
    20 & Uniform\\
    21 & Victor\\
    22 & Whiskey\\
    23 & X-ray\\
    24 & Yankee\\
    25 & Zulu\\
\end{longtable}
\end{lstlisting}


\section{Code for missing graphic fix me example}
\label{code-missing-graphic-fix-me-example}

\subsection{Missing rectangle graphic}
\begin{lstlisting}[language=tex]
\begin{figure}
    \centering
    \fxgraphicrectwarning{Add missing rectangle graphic}
    
    \caption{A missing rectangle graphic example}
\end{figure}
\end{lstlisting}

\subsection{Missing square graphic}
\begin{lstlisting}[language=tex]
\begin{figure}
    \centering
    \fxgraphicsqrewarning{Add missing square graphic}
    
    \caption{A missing square graphic example}
\end{figure}
\end{lstlisting}